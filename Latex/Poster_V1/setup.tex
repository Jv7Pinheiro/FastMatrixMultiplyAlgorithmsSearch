%% ------------------------------------------------------------
%% PACKAGES
%% ------------------------------------------------------------

%% For \circledast
\usepackage{amssymb,amsfonts,amsmath}

%% For \mathscr
\usepackage[mathscr]{eucal}

%% For \llbracket and \rrbracket, \varoast, \varoslash
\usepackage{stmaryrd}

%% For \boldsymbol
\usepackage{amsbsy}

%% For \bm (bold math)
\usepackage{bm}

%% For \set, \Set
\usepackage{braket}

%% Algorithms
\usepackage{algorithm,algpseudocode}
\renewcommand{\algorithmicprocedure}{\textbf{function}}

%% Cleveref - Must be loaded *after* algorithm and hyperref.
\usepackage[nameinlink]{cleveref}
\crefname{algorithm}{Alg.\@}{Algs.\@}
\crefname{figure}{Fig.\@}{Figs.\@}
\crefname{table}{Tab.\@}{Tabs.\@}
\crefname{section}{Sec.\@}{Secs.\@}	

\usepackage{subfig}
\usepackage{tikz}
\usetikzlibrary{3d,calc}

%% ------------------------------------------------------------
%% MACROS
%% ------------------------------------------------------------


%% --- Extras ---
% Transpose
\newcommand{\Tra}{{\sf T}} 
\newcommand{\parens}[1]{(#1)}
\newcommand{\Parens}[1]{\left(#1\right)}
\newcommand{\dsquare}[1]{\llbracket #1 \rrbracket}
\newcommand{\Dsquare}[1]{\left\llbracket #1 \right\rrbracket}
\newcommand{\curly}[1]{\{ #1 \}}
\newcommand{\Curly}[1]{\left\{ #1 \right\}}
\newcommand{\Real}{\mathbb{R}}
\newcommand{\qtext}[1]{\quad\text{#1}\quad}
\newcommand{\iddots}{\scalebox{-1}[1]{$\ddots$}}
\newcommand{\plusequals}{\mathrel{+}=}
\DeclareMathOperator*{\argmin}{arg\,min}

%% --- Vectors ---
% vector
\newcommand{\V}[2][]{{\bm{#1\mathbf{\MakeLowercase{#2}}}}} 
% element of vector
\newcommand{\VE}[3][]{#1{\MakeLowercase{#2}}_{#3}} 
% vector in series
\newcommand{\Vn}[3][]{{\bm{#1\mathbf{\MakeLowercase{#2}}}}^{(#3)}} 
% transposed vector in series
\newcommand{\VnTra}[3][]{{\bm{#1\mathbf{\MakeLowercase{#2}}}}^{(#3)\Tra}} 
% element of vector in series
\newcommand{\VnE}[4][]{#1{\MakeLowercase{#2}}^{(#3)}_{#4}} 

%% --- Matrices ---
% matrix
\newcommand{\M}[2][]{{\bm{#1\mathbf{\MakeUppercase{#2}}}}} 
% matrix in series
\newcommand{\Mn}[3][]{{\bm{#1\mathbf{\MakeUppercase{#2}}}}^{(#3)}} 
% transposed matrix in series 
\newcommand{\MnTra}[4][]{{\bm{#1\mathbf{\MakeUppercase{#2}}}}^{(#3)\Tra}} 
% matrix column
\newcommand{\MC}[3][]{\V[#1]{#2}_{#3}} 
% column of matrix in series
\newcommand{\MnC}[4][]{\Vn[#1]{#2}{#3}_{#4}} 
% transposed column of matrix in series
\newcommand{\MnCTra}[4][]{\VnTra[#1]{#2}{#3}_{#4}} 
% matrix element
\newcommand{\ME}[3][]{#1{\MakeLowercase{#2}}_{#3}} 
% element of matrix in series
\newcommand{\MnE}[4][]{#1{\MakeLowercase{#2}}^{(#3)}_{#4}} 

%% --- Tensors ---
% tensor
\newcommand{\T}[2][]{\boldsymbol{#1\mathscr{\MakeUppercase{#2}}}} 
% tensor slide
\newcommand{\TS}[3][]{\M[#1]{#2}_{#3}}
% tensor element
\newcommand{\TE}[3][]{#1{\MakeLowercase{#2}}_{#3}}
% matriczied tensor
\newcommand{\Mz}[3][]{\M[#1]{#2}_{(#3)}}
% tensor in series
\newcommand{\Tn}[3][]{{\boldsymbol{#1\mathscr{\MakeUppercase{#2}}}}^{(#3)}} 
% matricized tensor in series
\newcommand{\Mzn}[4][]{#1\mathbf{\MakeUppercase{#2}}^{(#3)}_{(#4)}}

%% --- Operators ---
% outer product
\newcommand{\Oprod}{\circ} 
% Kronecker product
\newcommand{\Kron}{\otimes} 
% Khatri-Rao product
\newcommand{\Khat}{\odot} 
% Hadamard (elementwise multiply)
\newcommand{\Hada}{\ast} 
\newcommand{\BigHada}{\mathop{\mbox{\fontsize{18}{19}\selectfont $\circledast$}}} 
% Elementwise divide
\newcommand{\Divi}{\varoslash}

\newcommand{\GB}[1]{{\color{blue}\textbf{GB:}~#1}}
\newcommand{\ignore}[1]{}

